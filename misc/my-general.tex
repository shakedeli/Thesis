% General-purpose definitions and inclusions
% you are using in any document 
% (regardless of its class and style files used),
% e.g. package uses:

\usepackage{booktabs}   %% For formal tables:
                        %% http://ctan.org/pkg/booktabs
\usepackage{subcaption} %% For complex figures with subfigures/subcaptions
                        %% http://ctan.org/pkg/subcaption
\usepackage{xspace}
\usepackage{libertine}
\usepackage{balance}

% \usepackage[utf8]{inputenc}
\usepackage{enumitem}
\usepackage{listings}
\usepackage{amsmath}
\usepackage{dsfont}
\usepackage{amssymb}


%% Colors 
\usepackage{xcolor}

\usepackage[]{todonotes}

%% Figures and Graphs Packages
\usepackage{graphicx}% http://ctan.org/pkg/graphicx
\usepackage{caption}
\usepackage{float}
% % \usepackage{placeins} 
% \usepackage{flafter}
% \usepackage{wrapfig}
% \graphicspath{ {./graphics/sequential/}{./graphics/algorithm/}{./graphics/graphs/}{./graphics/} }

%https://tex.stackexchange.com/questions/497228/automatically-convert-a-number-to-use-the-correct-si-unit-prefix
\usepackage{siunitx}
\ExplSyntaxOn
% Internals
\int_new:N \l__nebu_min_prefix_int
\tl_new:N \l__nebu_base_number_tl
\tl_new:N \l__nebu_mode_tl
\cs_new:Npn \nebu_prefix:n #1
  {
    \exp_args:Nf \__nebu_prefix:n
      { \exp_args:Nf \fp_to_scientific:n { \tl_lower_case:n {#1} } }
  }
\cs_new:Npn \__nebu_prefix:n #1
  { \__nebu_prefix:nwnw #1 \q_stop }
\cs_new:Npn \__nebu_prefix:nwnw #1 e #2 \q_stop
  { \__nebu_find_prefix:nnn {#2} {#1} {1} }
\cs_new:Npn \__nebu_find_prefix:nnn #1 #2 #3
  {
    \tl_if_exist:cT { l__nebu_ #1 \tl_use:N \l__nebu_mode_tl _prefix_tl }
      {
        \use_i_delimit_by_q_stop:nw
          { \__nebu_output:nnn {#2} {#1} {#3} }
      }
    \int_compare:nNnT {#1} < \l__nebu_min_prefix_int
      {
        \use_i_delimit_by_q_stop:nw
          {
            \exp_args:Nf \__nebu_find_prefix:nnn
              { \int_eval:n { \l__nebu_min_prefix_int } } {#2}
              { #3 / 1\prg_replicate:nn { \l__nebu_min_prefix_int - #1 }{ 0 } }
          }
      }
    \use_i:nn
      {
        \exp_args:Nf \__nebu_find_prefix:nnn
          { \int_eval:n {#1-1} } {#2} {#3*10}
      }
      \q_stop
  }
\exp_args_generate:n { fv }
\cs_new:Npn \__nebu_output:nnn #1 #2 #3
  {
    \exp_args:Nfv \nebu_output:nn
      { \fp_to_decimal:n {#1*#3} } { l__nebu_ #2 \tl_use:N \l__nebu_mode_tl _prefix_tl }
  }
\cs_new_protected:Npn \nebu_prefix_set:Nnn #1 #2 #3
  {
    \exp_args:Nxx \__nebu_prefix_set:nnN
      { \tl_trim_spaces:n {#2} } { \int_eval:n {#3} } #1
  }
\cs_new_protected:Npn \__nebu_prefix_set:nnN #1 #2 #3
  {
    \tl_clear_new:c { l__nebu_ #2 _prefix_tl }
    \tl_clear_new:c { l__nebu_ #2 _siunitx_prefix_tl }
    \tl_set:cn { l__nebu_ #2 _prefix_tl } {#1}
    \tl_set:cn { l__nebu_ #2 _siunitx_prefix_tl } {#3}
    \int_set:Nn \l__nebu_min_prefix_int { \int_min:nn {#2} { \l__nebu_min_prefix_int } }
  }
% User interface
\NewExpandableDocumentCommand \prefix { m }
  { \nebu_prefix:n {#1} }
\cs_new:Npn \nebu_output:nn #1 #2 { #1\, \textrm{#2} }
\NewDocumentCommand \setprefix { m m m }
  { \nebu_prefix_set:Nnn #1 {#2} {#3} }
\DeclareSIPrefix \none { } { 0 }
\setprefix \none { } { 0 }
%
\cs_new_protected:Npn \__nebu_store:nn #1 #2
  {
    \tl_set:Nn \l__nebu_base_number_tl {#1}
    \cs_set:Npn \prefix {#2}
  }
\NewExpandableDocumentCommand \prefixSI { o m o m }
  {
    \group_begin:
      \cs_set_eq:NN \nebu_output:nn \__nebu_store:nn
      \tl_set:Nn \l__nebu_mode_tl { _siunitx }
      \nebu_prefix:n {#2}
      \SI [#1] { \l__nebu_base_number_tl } [#3] {#4}
    \group_end:
  }
\ExplSyntaxOff

\setprefix \yocto { y } { -24 }
\setprefix \zepto { z } { -21 }
\setprefix \atto  { a } { -18 }
\setprefix \femto { f } { -15 }
\setprefix \pico  { p } { -12 }
\setprefix \nano  { n } { -9 }
\setprefix \micro { \SIUnitSymbolMicro } { -6 }
\setprefix \milli { m } { -3 }
\setprefix \centi { c } { -2 }
\setprefix \deci  { d } { -1 }
\setprefix \deca  { da } { 1 }
\setprefix \hecto { h }  { 2 }
\setprefix \kilo  { k }  { 3 }
\setprefix \mega  { M }  { 6 }
\setprefix \giga  { G }  { 9 }
\setprefix \tera  { T }  { 12 }
\setprefix \peta  { P }  { 15 }
\setprefix \exa   { E }  { 18 }
\setprefix \zetta { Z }  { 21 }
\setprefix \yotta { Y }  { 24 }






% and macros/command defintions:
\newcommand{\mysketch}{Quancurrent\xspace}

% %% Pseudo-code: 
% \usepackage{algpseudocode}
\newcommand{\myvar}[1]{$\mathit{#1}$}
\newcommand{\var}[1]{\mathit{#1}}

% \algrenewcommand{\algorithmiccomment}[1]{{$\triangleright$}{#1}}
% \algnewcommand{\LongComment}[1]{$\triangleright$ \begin{minipage}[t]{}#1\strut\end{minipage}}

%Initializing counter to add continuously line numbers to algorithms
\newcounter{mycounter}
%Start of alg - \setcounter{ALG@line}{\value{mycounter}}
%End of alg - \setcounter{mycounter}{\value{ALG@line}}
\setcounter{mycounter}{0}



% \algrenewcommand{\algorithmiccomment}[1]{\hfill\triangleright$ \eqparbox{}{#1}}
% \algnewcommand{\LongComment}[1]{\hfill$\triangleright$ \begin{minipage}[t]{\eqboxwidth{}}#1\strut\end{minipage}}
% \usepackage[ruled,linesnumbered,noresetcount]{algorithm2e}
% \AtBeginEnvironment{algorithm}{\SetArgSty{textrm}} 
% \newcommand\mycommfont[1]{\rmfamily{#1}}
% \SetCommentSty{mycommfont}
% \newcommand{\myvar}[1]{$\mathit{#1}$}
% \newcommand{\var}[1]{\mathit{#1}}
% \SetKwComment{Comment}{$\triangleright$\ }{}
% \SetEndCharOfAlgoLine{}
% \SetKwProg{Procedure}{Procedure}{:}{end}
% \SetKwRepeat{Repeat}{do}{while}

%Self-defined math commands 
\newcommand{\s}{$\sigma$\xspace}
\newcommand{\fs}{\(f(\sigma)\)\xspace}
\newcommand{\fstag}{\(f(\sigma')\)\xspace}
\newcommand{\ls}{\(l(\sigma)\)\xspace}
\renewcommand{\S}{\(\mathnormal{S}\)\xspace}
\newcommand{\Q}{\(\mathnormal{Q}\)\xspace}
\renewcommand{\O}{\(\mathnormal{O}\)\xspace}
\newcommand{\A}{$A=\mathcal{S}(f(\sigma))$\xspace}
\renewcommand{\d}{\(\delta\)\xspace}


%colors
\newcommand{\inred}[1]{{\color{red} #1 }}
\newcommand{\ingray}[1]{{\color{gray} #1 }}
\newcommand{\indarkgray}[1]{{\color{darkgray} #1 }}
\newcommand{\inblue}[1]{{\color{blue} #1 }}
\newcommand{\todoredefined}[2][]{\todo[color=red, #1]{#2}}
%Example:
%\todoredefined[color=green]{Test of newly defined command, requesting a green color.}




% bibliography 
\bibliographystyle{ACM-Reference-Format}


% HOW TO:

% 1) Pseudocode: 
% \begin{algorithm}[h]
% \caption{} \label{alg:}
% \begin{algorithmic}[1] 

% \end{algorithmic}
% \end{algorithm}


% \keepUnusedAbbreviations