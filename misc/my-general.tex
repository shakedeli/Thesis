% General-purpose definitions and inclusions
% you are using in any document 
% (regardless of its class and style files used),
% e.g. package uses:

\usepackage{booktabs}   %% For formal tables:
                        %% http://ctan.org/pkg/booktabs
\usepackage{subcaption} %% For complex figures with subfigures/subcaptions
                        %% http://ctan.org/pkg/subcaption
\usepackage{xspace}
\usepackage{libertine}
\usepackage{balance}

% \usepackage[utf8]{inputenc}
\usepackage{enumitem}
\usepackage{listings}
\usepackage{amsmath}
\usepackage{dsfont}

%% Colors 
\usepackage{xcolor}

\usepackage[]{todonotes}

%% Figures and Graphs Packages
\usepackage{graphicx}% http://ctan.org/pkg/graphicx
\usepackage{caption}
\usepackage{float}
% % \usepackage{placeins} 
% \usepackage{flafter}
% \usepackage{wrapfig}
% \graphicspath{ {./graphics/sequential/}{./graphics/algorithm/}{./graphics/graphs/}{./graphics/} }

% and macros/command defintions:
\newcommand{\mysketch}{Quancurrent\xspace}

% %% Pseudo-code: 
% \usepackage{algpseudocode}
\newcommand{\myvar}[1]{$\mathit{#1}$}
\newcommand{\var}[1]{\mathit{#1}}

% \algrenewcommand{\algorithmiccomment}[1]{{$\triangleright$}{#1}}
% \algnewcommand{\LongComment}[1]{$\triangleright$ \begin{minipage}[t]{}#1\strut\end{minipage}}

%Initializing counter to add continuously line numbers to algorithms
\newcounter{mycounter}
%Start of alg - \setcounter{ALG@line}{\value{mycounter}}
%End of alg - \setcounter{mycounter}{\value{ALG@line}}
\setcounter{mycounter}{0}



% \algrenewcommand{\algorithmiccomment}[1]{\hfill\triangleright$ \eqparbox{}{#1}}
% \algnewcommand{\LongComment}[1]{\hfill$\triangleright$ \begin{minipage}[t]{\eqboxwidth{}}#1\strut\end{minipage}}
% \usepackage[ruled,linesnumbered,noresetcount]{algorithm2e}
% \AtBeginEnvironment{algorithm}{\SetArgSty{textrm}} 
% \newcommand\mycommfont[1]{\rmfamily{#1}}
% \SetCommentSty{mycommfont}
% \newcommand{\myvar}[1]{$\mathit{#1}$}
% \newcommand{\var}[1]{\mathit{#1}}
% \SetKwComment{Comment}{$\triangleright$\ }{}
% \SetEndCharOfAlgoLine{}
% \SetKwProg{Procedure}{Procedure}{:}{end}
% \SetKwRepeat{Repeat}{do}{while}

%Self-defined math commands 
\newcommand{\s}{$\sigma$\xspace}
\newcommand{\fs}{\(f(\sigma)\)\xspace}
\newcommand{\fstag}{\(f(\sigma')\)\xspace}
\newcommand{\ls}{\(l(\sigma)\)\xspace}
\renewcommand{\S}{\(\mathnormal{S}\)\xspace}
\newcommand{\Q}{\(\mathnormal{Q}\)\xspace}
\renewcommand{\O}{\(\mathnormal{O}\)\xspace}
\newcommand{\A}{$A=\mathcal{S}(f(\sigma))$\xspace}
\renewcommand{\d}{\(\delta\)\xspace}


%colors
\newcommand{\inred}[1]{{\color{red} #1 }}
\newcommand{\ingray}[1]{{\color{gray} #1 }}
\newcommand{\indarkgray}[1]{{\color{darkgray} #1 }}
\newcommand{\inblue}[1]{{\color{blue} #1 }}
\newcommand{\todoredefined}[2][]{\todo[color=red, #1]{#2}}
%Example:
%\todoredefined[color=green]{Test of newly defined command, requesting a green color.}



% bibliography 
\bibliographystyle{ACM-Reference-Format}


% HOW TO:

% 1) Pseudocode: 
% \begin{algorithm}[h]
% \caption{} \label{alg:}
% \begin{algorithmic}[1] 

% \end{algorithmic}
% \end{algorithm}