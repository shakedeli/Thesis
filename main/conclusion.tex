\chapter{Conclusion and open questions}
\label{chap:conclusion}



We presented Quancurrent, a concurrent scalable Quantiles sketch. We have evaluated it and shown it to be linearly scalable for both updates and queries while providing accurate estimates, i.e., retaining a small error bound with reasonable query freshness. Moreover, it achieves higher performance than state-of-the-art concurrent quantiles solutions with better query freshness.

Quancurrent's scalability arises from allowing multiple threads to engage concurrently in merge-sorts, which are a sequential bottleneck in previous solutions. We dramatically reduce the synchronization overhead by accommodating occasional data races that cause samples to be duplicated or dropped, a phenomenon we refer to as holes. This approach leverages the observation that sketches are approximate to begin with, and so the impact of such holes is marginal.

Future work may leverage this observation to achieve high scalability in other sketches or approximation algorithms.

Another direction for future work is Quantiles sketch-based index. Index structures are used when efficient data access is needed. Data structures that support efficient data search and efficient data access are critical in practical settings where the large amounts of underlying data are usually paired with high search volumes and with high amounts of concurrency on the hardware side via tens or even hundreds of parallel threads. 

Range index structures, like B-Trees, "predict" the location of value within a key sorted set. It is common for the index always to be stored in the main memory, with the data itself sitting on the disk. Quantile sketches can summarize large amounts of data in sub-linear space complexity. Therefore, We can use Quantiles sketches to reduce the memory overhead of an index. 

An index structure was recently proposed by Brown et al.~\cite{cist}. They implemented a non-blocking concurrent interpolation search tree. They designed a parallel rebuilding algorithm to provide a fast reconstruction of subtrees. A Quantiles sketch may be used to reduce the number of rebalancing operations by tracking the imbalance of the tree. The Quantiles sketch will approximate the distribution of keys in each subtree and will be queried to check if rebuilding is needed. 

