\chapter{Analysis}
\label{chap:analysis}

In Section~\ref{sec:holes-analysis} we analyze the expected number of holes, and in Section~\ref{sec:error-analysis} we analyze \mysketch's error.

\section{Holes Analysis}
\label{sec:holes-analysis}

Because the update operation moves elements from a thread's local buffer to a shared G\&SBuffer non-atomically, holes may occur when the owner thread reads older elements that were written to the shared buffer in a previous window. The missed (delayed) writes may later overwrote newer writes. Together, for each hole, an old value is duplicated and a new value is dropped. As such, we created a dependency between samples because we dropped an independent sample and gave double weight to another. 

We analyze the expected number of holes under the assumption of a \emph{uniform stochastic scheduler}~\cite{alistarh2016lock}, which schedules each thread with a uniform probability in every step. That is, at each point in the execution, the probability for each thread to take the next step is $\frac{1}{N}$. Note that holes are random and distributed from the same distribution. Therefore, they do not affect the samples' mean and only affect the accuracy of estimation. Below we show that the expected number of holes is fairly small and that they have a marginal effect on the estimation accuracy. 

Denote by $H$ the total number of holes in some batch of $2k$ elements. G\&SBuffer's array is divided into $\frac{2k}{b}$ regions, each consisting of $b$ slots populated by the same thread. Denote by $H_1,\dots, H_{\frac{2k}{b}}$ the number of holes in regions $1, \dots, \frac{2k}{b}$, respectively.

The slots in region $j$ are written to by the thread that successfully increments the shared index from $(j-1)b$ to $jb$. We refer to this thread as $T_j$. Note that multiple regions may have the same writing thread. The shared G\&SBuffer's owner, $T_O$, is $T_{\frac{2k}{b}}$. To initiate a batch update, $T_O$ creates a local copy of his G\&SBuffer by iteratively reading the array. A hole is read in some region $j$ if $T_O$ reads some index $i+1$ in this region before the writer thread $T_j$ writes to the corresponding index in the same region.

\paragraph*{Analysis of $\boldsymbol{H_j}$.} When $T_O$ increments the index from $2k-b$ to $2k$, $T_j$ may have completed any number of writes between $0$ and $b$ to region $j$. We first consider the case that $T_j$ hasn't completed any writes. In this case, for a hole to be read in slot $i+1$ of region $j$, $T_O$'s read of slot $i+1$ must overtake $T_j$'s write of the same slot. To this end, $T_O$ must write $b$ values (from its own local buffer), read $(j-1)b$ values from the first $j-1$ regions and then read values from slots $1,\dots,i+1$ in this region (a total of $b+(j-1)b+i+1$ steps), before $T_j$ takes $i+1$ steps. The probability that $T_O$ reads a hole in region $j$ for the first time, in slot $i+1$ is:
\begin{align}
\pi_{i,j} \triangleq P[\text{hole in slot }i+1  &\mid \text{no hole in slots } 1\dots i] \\
                                            & \cdot P[\text{no hole in slots } 1\dots i].
\end{align}
For a hole to be read in slot $i+1$ of region $j$ (not necessarily for the first time in this region), $T_O$ must take $b+(j-1)b+i+1$ steps while $T_j$ takes at most $i$ steps, with $T_O$'s read of slot $i+1$ being last.
But if $T_j$ takes fewer than $i$ steps, a hole is necessarily read earlier than slot $i+1$. Therefore, we can bound $\pi_{i,j}$ by considering the probability that $T_j$ takes exactly $i$ steps while $T_O$ takes $b+(j-1)b+i$ steps, and then $T_O$ takes a step. Ignoring steps of other threads, each of $T_j$ and $T_O$ has a probability of $\frac{1}{2}$ to take a step before the other. Therefore,
\begin{align}
\pi_{i,j} \leq \left(\frac{1}{2}\right)^{jb + 2i +1} {{jb+2i} \choose i}. \label{eq:pi-i-j}
\end{align}
Note that this includes schedules in which $T_O$ reads holes in previous slots in the same region, therefore it is an upper bound.
Given that $T_j$ has not yet written in region $j$, the probability, $p_j$, that $T_O$ reads at least $1$ hole in region $j$ is bounded as follows:
\begin{align}
    p_j &\leq \sum_{i=0}^{b-1} \pi_{i,j}. \label{eq:p-j}
    %&\leq \sum_{i=0}^{b-1} \left(\frac{1}{2}\right)^{2b + jb -1} {{2b+jb-2} \choose b-1}
\end{align}
% In the supplementary material we prove that the summed sequence is monotonically increasing, therefore:
% \[P_1 \leq b \cdot \left(\frac{1}{2}\right)^{3b-1} {{3b-2} \choose b-1}.\]
In the supplementary material (Claim \ref{claim:holes-pi-mono}) we prove that the above upper bound of $\pi_{i,j}$ (Equation~\ref{eq:pi-i-j}) is monotonically increasing for $i\in {0,1,\dots,b-1}$ and for $j\in \mathds{N}$. Therefore, 
\begin{align}
\pi_{i,j} \leq \left(\frac{1}{2}\right)^{jb + 2b -1} {{jb+2b-2} \choose b-1}. \label{eq:bound-pi-i-j}
\end{align}
Using Equation~\ref{eq:p-j},
\begin{align}
    p_j &\leq \sum_{i=0}^{b-1} \left(\frac{1}{2}\right)^{jb + 2b -1} {{jb+2b-2} \choose b-1}\\
    &\leq b \cdot \left(\frac{1}{2}\right)^{jb + 2b -1} {{jb+2b-2} \choose b-1} \label{eq:bound-p-j}
\end{align}
If $T_j$ has completed any number of writes to region $j$, the probability that $T_O$ reads holes is even lower.
%
Therefore, the probability that $H_j \geq 1$ is bounded from above by $p_j$.
Using this, we bound the expected total number of holes in region $j$:
\begin{align}
E\left[H_j\right]= P(H_j=0)\cdot 0 + P(H_j=1)\cdot 1 + \dots + P(H_j=b)\cdot b.
\end{align}
$T_O$ can read at most $b$ holes, therefore,
\begin{flalign}
E\left[H_j\right] &< b\cdot \left(P(H_j=1) + \dots + P(H_j=b) \right) \\
                    &= b\cdot P(H_j\geq1) < b\cdot p_j. \label{eq:bound-E-H-j}
\end{flalign}
Next we show that $E[H_1]\leq1.4$ for all $b$.
\begin{lemma}
$E[H_1] \leq 1.4$ for all $b \in \mathds{N}$.
\label{lemma:holes-H1}
\end{lemma}
\begin{proof}
Denote 
\begin{align}
    g(b) \triangleq b^2 \cdot \left(\frac{1}{2}\right)^{3b -1} {{3b-2} \choose b-1}
\end{align}
From Equation~\ref{eq:bound-E-H-j}, $E[H_1]$ is bounded by 
\begin{align}
    E[H_1] \leq b^2 \cdot \left(\frac{1}{2}\right)^{3b -1} {{3b-2} \choose b-1} = g(b)
    \label{eq:bound-E-H-1}
\end{align}
First, we show that $g(b)$ is monotonically decreasing for $b \geq 12$.
\begin{align}
    \frac{g(b+1)}{g(b)} &= \frac{(b+1)^2 \cdot \left(\frac{1}{2}\right)^{3b +2} {{3b+1} \choose b}}{b^2 \cdot \left(\frac{1}{2}\right)^{3b -1} {{3b-2} \choose b-1}} \\
    &= \left(\frac{1}{2}\right)^3 \cdot \left(\frac{3}{2}\right) \frac{(b+1)^2}{b^2} \cdot \frac{3b-1}{b} \cdot \frac{3b+1}{2b+1}
\end{align}
 In the Supplementary Material (Claims~\ref{claim:g-1}-\ref{claim:g-3}), we show that for $b>12$
\begin{align}
     \frac{g(b+1)}{g(b)} &= \left(\frac{1}{2}\right)^3 \cdot \left(\frac{3}{2}\right) \underbrace{\frac{(b+1)^2}{b^2}}_{\leq \left(\frac{13}{12}\right)^2} \cdot \underbrace{\frac{3b-1}{b}}_{<3} \cdot \underbrace{\frac{3b+1}{2b+1}}_{<\frac{3}{2}} < 1
 \end{align}
Therefore, $g(b)$ is monotonically decreasing for $b \geq 12$.
Lastly,
\begin{align}
    \max_{1\leq b \leq 12}\{f(b)\} = f(9)=1.305 < 1.4.
\end{align}
That is, for all $b \in \mathds{N}$,
\begin{align}
    g(b) < 1.4.
    \label{eq:bound-g-b}
\end{align}
From Equation~\ref{eq:bound-E-H-1} and Equation~\ref{eq:bound-g-b},
\[\forall b\in \mathds{N}, \; E[H_{1}] \leq 1.4.\]
\end{proof}
\begin{lemma} \label{lem:holes-Hj}
If $E[h_j] \leq \alpha_{j}$, then $E[h_{j+1}] \leq \frac{1}{2}\alpha_{j}$ for all $b\in \mathds{N}$ and $j\geq1, j\in \mathds{N}$. 
\end{lemma}
\begin{proof}
From Equation~\ref{eq:bound-E-H-j},
\begin{align}
&E[H_j] \leq b^2 \left(\frac{1}{2}\right)^{bj+2b -1} \cdot {{jb+2b-2} \choose {b-1}}. \\
&E[H_{j+1}] \leq b^2 \left(\frac{1}{2}\right)^{bj+3b -1} \cdot {{jb+3b-2} \choose {b-1}}. 
\end{align}
Denote
\begin{align}
        \alpha_j \triangleq b^2 \left(\frac{1}{2}\right)^{bj+2b -1} \cdot {{jb+2b-2} \choose {b-1}}.
\end{align}
We show that $E[H_{j+1}] \leq \frac{1}{2}\alpha_j$.
\begin{align}
&& E[H_{j+1}] &\leq b^2 \left(\frac{1}{2}\right)^{bj+3b -1} \cdot {{jb+3b-2} \choose {b-1}} \\ &&              &= \frac{1}{2} \cdot b^2 \left(\frac{1}{2}\right)^{bj+2b -1} \cdot \left(\frac{1}{2}\right)^{b-1} {{jb+3b-2} \choose {b-1}} \\
\text{Using Claim~\ref{claim:E-H-j-help}}&&              &\leq \frac{1}{2} \cdot b^2 \left(\frac{1}{2}\right)^{bj+2b -1} \cdot {{jb+2b-2} \choose {b-1}} \\
&&              &= \frac{1}{2}\alpha_j.
\end{align}
Therefore, $E[H_{j+1}] \leq \frac{1}{2}\alpha_j$.
\end{proof}
Using the linearity of expectation, we bound the expected number of holes in a batch:
\[E\left[H\right]= E\left[H_1\right] + E\left[H_2\right] + \dots + E\left[H_{\frac{2k}{b}}\right]. \]
From Lemma~\ref{lemma:holes-H1} and Lemma~\ref{lem:holes-Hj}, 
\begin{align}
    E[H] &= \sum_{j=1}^{\frac{2k}{b}} E[H_j] \\
        &\leq  \sum_{j=1}^{\frac{2k}{b}} \left(\frac{1}{2}\right)^{j-1}\cdot E[H_1] \\
        &\leq  \sum_{j=1}^{\infty} \left(\frac{1}{2}\right)^{j-1}\cdot E[H_1] \\
        &\leq  \underbrace{\sum_{j=1}^{\infty} \left(\frac{1}{2}\right)^{j-1}}_{<2}\cdot 1.4 \\
        &\leq 2\cdot 1.4 = 2.8
\end{align}
Together, this implies that  $E[H] \leq 2.8$ for all $b \in \mathds{N}$.

\section{Error Analysis}
\label{sec:error-analysis}

The source of \mysketch's estimation error is twofold: (1) the error induced by sub-sampling the stream, and (2) the additional error induced by concurrency. For the former, we leverage the existing literature on analysis of sequential sketches. We analyze the latter. 
%For $b=16$ and $N=32$, we have shown that the expected number of holes is less than $1$. The empirical evaluation on Section~\ref{sec:eval} confirms that this is reasonably true in the general case. Thus, we disregard the effect of holes on the error analysis.
%As the expected number of holes is less than $1$, we disregard their effect on the error analysis. 
As the expected number of holes is fairly small and the holes are random, we disregard their effect on the error analysis. 

First, our buffering mechanism induces a relaxation. Let $S$ be the number of NUMA nodes. Recall that each NUMA node has a Gather\&Sort object that contains two buffers of size $2k$. In addition, each of the $N$ update threads has a local buffer. When the G\&SBuffer is full, the local buffer of the owner is empty so at most $N-S$ threads might have locally buffered elements. Therefore, the buffering relaxation $r$ is $4kS+(N-S)b$.


Rinberg et al.~\cite{Rinberg_2020_fast_sketches} show that for a query of a $\phi$-quantile, an $r$-relaxation of a Quantiles sketch with parameters $\epsilon_c$ and $\delta_c$, returns an element, $x$, such that
\[ rank(x) \in [(\phi-\epsilon_r)n,(\phi+\epsilon_r)n] \]
with probability at least $(1-\delta_c)$, for $\epsilon_r=\epsilon_c+\frac{r}{n}(1-\epsilon_c)$.

On top of this relaxation, our cache mechanism induces further staleness. Here, the staleness depends on $\rho$. Let $n_{old}$ be the stream size of the cached snapshot, and let $n_{new}$ be the current stream size. If $n_{new} / n_{old} \leq \rho$ then the query is answered from the cached snapshot.
Denote $\rho \triangleq 1+\epsilon'$ for some $\epsilon' \geq 0 $. The rank of the element returned by the cached snapshot is in the range:
% \abovedisplayskip=5pt
% \abovedisplayshortskip=0pt
% \belowdisplayskip=5pt
% \belowdisplayshortskip=0pt
\begin{equation}
\left[\left(\phi-\epsilon_r\right)n_{old},\left(\phi+\epsilon_r\right)n_{old}\right]
\end{equation}
As $n_{old} \leq n_{new}$ and $\epsilon' \geq 0 $ then, 
\begin{align}
\left(\phi+\epsilon_r\right)n_{old} \leq \left(\phi+\epsilon_r\right)n_{new} \leq \left(\phi+\left(\epsilon'+\epsilon_r\right)\right)n_{new} &&
\end{align}
On the other hand, as $n_{old} \geq \frac{n_{new}}{\rho}$,
\begin{align}
\left(\phi-\epsilon_r\right)n_{old} \geq &\left(\phi-\epsilon_r\right)\frac{n_{new}}{\rho}= &&\text{($\rho = 1+\epsilon'$)} \\
&\left(\frac{\phi}{1+\epsilon'}-\frac{\epsilon_r}{1+\epsilon'}\right)n_{new}= && \\
&\left(\frac{\phi+\phi\epsilon'-\phi\epsilon'}{1+\epsilon'}-\frac{\epsilon_r}{1+\epsilon'}\right)n_{new}= && \\
&\left(\frac{\phi(1+\epsilon')}{1+\epsilon'}-\frac{\phi\epsilon'}{1+\epsilon'}-\frac{\epsilon_r}{1+\epsilon'}\right)n_{new}= &&\text{($0\leq \frac{\phi}{1+\epsilon'} \leq \frac{1}{1+\epsilon'}$)} \\
&\left(\phi-\frac{\epsilon'}{1+\epsilon'}-\frac{\epsilon_r}{1+\epsilon'}\right)n_{new} \geq && \\
&\left(\phi-\frac{1}{1+\epsilon'}\left(\epsilon'+\epsilon_r\right)\right)n_{new} \geq &&\text{($\frac{1}{1+\epsilon'} \leq 1$)} \\
%& \text{(Because $\phi \leq 1$, $\epsilon' \geq 0$ then, $\frac{\phi\epsilon'}{1+\epsilon'} \leq \frac{\epsilon'}{1+\epsilon'} \leq 1$)} &&\\
&\left(\phi-\left(\epsilon'+\epsilon_r\right)\right)n_{new} &&
\end{align}
Because $\phi \leq 1$ and $\epsilon' \geq 0$ then, $\frac{\phi\epsilon'}{1+\epsilon'} \leq \frac{\epsilon'}{1+\epsilon'} \leq 1$. Also $1+\epsilon' \geq 1$ then $\frac{1}{1+\epsilon'} \leq 1$.\\
% Therefore,
% \begin{flalign*}
% \geq\left(\phi-\left(\epsilon'+\epsilon_r\right)\right)n_{new}
% \end{flalign*}
Therefore, the query returns a value within the range
\begin{equation}
    \left[\left(\phi-\epsilon \right)n,\left(\phi+\epsilon \right)n\right]
\end{equation}
% \[\left[\left(\phi-\epsilon \right)n,\left(\phi+\epsilon \right)n\right]\] 
for $\epsilon \triangleq \epsilon_r + \epsilon'$.

% As the holes occur prior to the sketch sampling, we can model the estimated rank error by first applying the error induced by the holes, and then the error of the concurrent Quantiles sketch. This is depicted in Figure~\ref{fig:error_model}. Given a stream \(A = \{x_1, ... , x_n\}\), we refer to the stream with holes as \(A'\). On each batch of $2k$ elements, the number of holes is bounded by $0\leq h\leq (N-S)b$. In the appendix, we analyze the error induced by the holes. Given a stream $A'$ of size n, we calculate the summary size $k_h$, as a function of $\epsilon_h$ and $\delta_h$, where $\epsilon_h$ and $\delta_h$ are parameters of the error induced by the holes. We show that the estimation error due to the holes of a quantile is up to an additive error $\pm \epsilon_h n$, with probability at least $1-\delta_h$. Agarwal et al.~\cite{mergeables_summaries} showed that the error of a Quantiles sketch with summary size $k_s$ is $\pm \epsilon_s n$ with probability at least $1-\delta_s$, where $\epsilon_s$ and $\delta_s$ are parameters of the sketch and $n$ is the size of the stream. Such an error is induced by down-sampling the stream. 

% For parameters $\epsilon_h$, $\epsilon_s$ and $\delta_h$, $\delta_s$, given the original stream $A$, we show that \mysketch estimates the rank of an element to within $\pm (\epsilon_h+\epsilon_s) n$ with probability at least $(1-\delta_h)(1-\delta_s)$. We set the error parameter $\epsilon_c=\epsilon_h+\epsilon_s$ and show that the summary size $k_c$ is minimized for $\epsilon_s=\frac{\epsilon_h}{h}$. We choose $\delta_c$ such that $\delta_h=\delta_s$ and $(1-\delta_h)(1-\delta_s)=1-\delta_c$.




% We now discuss query freshness. There are two sources of staleness in \mysketch. First, due to buffering. Denote $S$ for the number of NUMA nodes, the state of the shared sketch (with N threads) might miss up to 4k elements, per NUMA node, in the shared G\&SBuffers and up to $(N-S)\cdot b$ elements in local buffers. We denote $r = 4kS+(N-S)b$.
% In the appendix we prove that if queries are answered by taking an atomic snapshot of the shared sketch (i.e., the levels), then the algorithm is strongly linearizable with respect to an r-relaxed sequential Quantiles sketch. Namely, a query may miss up to $r$ updates. 


% The second source of staleness is the caching of \emph{snapshot}. 