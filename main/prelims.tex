% \chapter{Preliminaries}
% \label{chap:prelims}

We consider a shared memory model, where a finite number of threads execute \emph{operations} on shared \emph{objects}. An operation consists of an \emph{invocation} and a matching \emph{response}. A \emph{history} \(\mathnormal{H}\) is a finite sequence of operation invocation and response steps. A history \(\mathnormal{H}\) defines a partial order \(\prec_\mathnormal{H}\) on operations: Given operations \(op\) and \(op'\), \(op \prec_H op'\) if and only if \(response(op)\) precedes \(invocation(op')\) in \(\mathnormal{H}\). Two operations that do not precede each other are \emph{concurrent}. In a \emph{sequential history}, there are no concurrent operations. An object is specified using a \emph{sequential specification} $\mathcal{H}$, which is the set of its allowed sequential histories.
An operation \emph{op} is \emph{complete} in a history \(\mathnormal{H}\) if both \(invocation(op)\) and its matching \(response(op)\) are in \(\mathnormal{H}\).
A \emph{linearization} of a concurrent history \(\mathnormal{H}\), is a sequential history $H'$ such that: (1) $H' \in \mathcal{H}$, (2) $H'$ contains all completed operations and possibly additional non-complete ones, after adding matching responses, and, (3) $\prec_{H'}$ extends $\prec_H$. A correctness condition for randomized algorithms \emph{strong linearizability}~\cite{strong_linearizability}, defined as follows: 

\begin{definition}[strong linearizability]
\label{def:strong_linearizability}
A function \(f\) mapping executions to histories is \emph{prefix-preserving} if for any two executions $\sigma$, $\sigma$', where $\sigma$ is a prefix of $\sigma$', $f(\sigma)$ is a prefix of $f(\sigma')$. 
An object A is \emph{strongly linearizable} if there is a prefix-preserving function \(f\) that maps every history \(\mathnormal{H}\) of $A$ to a linearization of \(\mathnormal{H}\).
\end{definition}

Our algorithm is randomized and we consider a \emph{weak adversary} that determines the scheduling without observing the coin-flips.

We adopt a flavor of \emph{relaxed semantics}, as defined in \cite{Henzinger_2013_Quantitative_Relaxation}:

\begin{definition}[$r$-relaxation] \label{def:r-relaxtion}
A sequential history $H$ is an \emph{$r$-relaxation} of a sequential history H' if H is comprised of all but at most $r$ of the invocations in $H'$ and their responses,
and each invocation in $H$ is preceded by all but at most $r$ of the invocations that precede the same invocation in $H'$.
The $r$-relaxation of a sequential specification $\mathcal{H}$ is the set of histories that have $r$-relaxations in $\mathcal{H}$:
\[{H}^r \triangleq \left\{ H' \mid \exists H \in \mathcal{H}: H \text{ is an } r\text{-relaxation of } H'\right\}. \]
\end{definition}
