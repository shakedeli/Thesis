% Use this file to create "glossary entries" for abbreviations and acronyms,
% and other notation. The entries defined here don't necessarily have to be 
% used in the thesis (but then you have to decide whether or not to display
% the unused entries).

% For this file to compile (and the example text in the main/prelims.tex file),
% the package glossaries-extra is required. It is automatically included unless
% the noabbrevs class option is used.

% The following will alter the style for typesetting abbreviations when using 
% the \gls command. Note you can also use multiple styles by categorizing 
% abbreviations; see the documentation for the glossaries-extras package at:
% https://ctan.org/pkg/glossaries-extra
%
%\setabbreviationstyle[acronym]{long-short-sc}
%
% If you're wondering why we're setting the seemingly-redundant "notation 
% category", that's a hack discussed here:
% https://tex.stackexchange.com/q/630541/5640 

% \newacronym[%
%   category=notation-category,%
%   description=``The Senate and People of Rome'']% The description does not appear anywhere by default
%   {spqr}% the key of the acronym (used with the \gls command for example)
%   {SPQR}% the short form of the acronym
%   {Senātus Populusque Rōmānus}% the long form of the acronym

% \newacronym[%
%   category=notation-category,%
% description=A technology used in data storage devices]%
%   {smart}{SMART}{Self-Monitoring, Analysis and Reporting Technology}

% \newacronym[%
%   category=notation-category,%
%   description=to build or produce something{,} rather than purchasing it ready-made or paying someone to make it]%
%   {DIY}{DIY}{do it yourself}

% \newacronym[%
%   category=notation-category,%
%   description=a four-letter acronym]
%   {etla}
%   {ETLA}
%   {extended three-letter acronym}

% \newabbreviation[%
%   type=notation,%
%   category=notation-category,%
%   description=]% This abbreviation has no description; only the abbreviation and the unabbreviated form will be shown
%   {aut}{Aut}{Automorphism group}

% \newglossaryentry{symb:c}{%
%   type=notation,%
%   category=notation-category,%
%   name=$c$,%
%   description=the speed of light%
% }

% \newglossaryentry{symb:a-b-closed}{%
%   type=notation,%
%   category=notation-category,%
%   name=\ensuremath{a \pm b},%
%   description=the closed interval \ensuremath{\left[a-b,a+b\right]}%
% }

% \newglossaryentry{supercali}{%
%   type=notation,%
%   category=notation-category,%
%   name=supercalifragilisticexpialidocious,
%   description=%
%     Atoning for being educable through delicate beauty;
%     Something to say when you have nothing to say.}

\newacronym[%
  category=notation-category,%
  description=]%
  {API} %the key of the acronym (used with the \gls command for example)
  {API} % the short form of the acronym
  {Application Programming Interface} % the long form of the acronym
  
\newacronym[%
  category=notation-category,%
  description= atomic operation which takes a memory address${\textit{,}}$ an expected value and a new value${\textit{,}}$ compares the actual value with the expected value and if they match${\textit{,}}$ atomically swap the actual value with the new value]%
  {CAS}{CAS}{Compare-And-Swap}

\newacronym[%
  category=notation-category,%
  description= atomic operation which takes two addresses${\textit{,}}$ two corresponding expected values and two new values as arguments${\textit{,}}$ and atomically updates both addresses if they both match their expected value]%
  {DCAS}{DCAS}{Double-Compare-Double-Swap}
    
\newacronym[%
  category=notation-category,%
  description= can read fields that are concurrently modifies by a DCAS]%
  {DCASREAD}{DCAS\_READ}{Wait-free read operation}
    
\newacronym[%
  category=notation-category,%
  description=]%
  {FA}{F\&A}{Fetch-And-Add}
  
  \newacronym[%
  category=notation-category,%
  description= $A$ is a PAC learning algorithm if for given $\epsilon>0$ and $\delta<1$ $A$ outputs a hypothesis $h$ that has an average error less than or equal to $\epsilon$ on the samples with probability at least $1-\delta$]%
  {PAC Learning}{PAC Learning}{Probably Approximately Correct Learning}

  \newacronym[%
  category=notation-category,%
  description= A PAC Quantiles sketch with parameters $\epsilon$ and $\delta$ return an element $x$ for query($\phi$) after $n$ updates such that $R(A{\textit{,}}x) \in {[}(\phi-\epsilon)n{\textit{,}}(\phi+\epsilon)n{]}$ with probability at least $1-\delta$]%
  {PACQS}{PAC Quantiles Sketch}{Probably Approximately Correct Quantiles Sketch}


  \newacronym[%
  category=notation-category,%
  description= framework proposed by Rinberg et al.~\cite{Rinberg_2020_fast_sketches} for building concurrent data sketches]%
  {FCDS}{FCDS}{Fast Concurrent Data Sketches}


  \newacronym[%
  category=notation-category,%
  description=]%
  {epsilon}{$\epsilon$}{Bound on the estimation error}
  
  \newacronym[%
  category=notation-category,%
  description=]%
  {delta}{$\delta$}{Bound on the failure probability}

  \newacronym[%
  category=notation-category,%
  description=]%
  {N}{$N$}{Number of update threads}
  
\newacronym[%
  category=notation-category,%
  description=]%
  {b}{$b$}{Size of threads' local buffer}

  \newacronym[%
  category=notation-category,%
  description=]%
  {S}{$S$}{Number of NUMA nodes}

  \newacronym[%
  category=notation-category,%
  description=]%
  {k}{$k$}{The sketch summary size}
  

  \newacronym[%
  category=notation-category,%
  description= memory design used in multiprocessing where the memory access time depends on the memory location relative to the processor]%
  {NUMA}{NUMA}{Non-Uniform-Memory-Access}

  \newacronym[%
  category=notation-category,%
  description=]%
  {cdf}{CDF}{Cumulative Distribution Function}

  \newacronym[%
  category=notation-category,%
  description=]%
  {holes}{Holes}{Occurrences where thread-local buffered elements are sporadically overwritten by others without being propagated${\textit{,}}$ and others are duplicated${\textit{,}}$ i${.}$e${.}{\textit{,}}$ propagated more than once }
  
\newacronym[%
  category=notation-category,%
  description= as long as this threshold does not exceed$\textit{,}$ the query is answered from the cached snapshot]%
  {rho}{$\rho$}{Threshold for rebuilding a snapshot}
  
      \newacronym[%
  category=notation-category,%
  description=]%
  {MAXLEVEL}{$\mathrm{MAX\_LEVEL}$}{Maximum number of levels in the sketch}

    \newacronym[%
  category=notation-category,%
  description= given a stream $A$ with $n$ elements the rank of some $x$ (not necessarily in $A$) is the number of elements smaller than $x$ in $A$]%
  {Rank}{$\mathrm{Rank(A\textit{, }x)}$}{Rank of an element $x$ in $A$}
  
      \newacronym[%
  category=notation-category,%
  description= given a stream $A$ with $n$ elements the $\phi$ quantile of $A$ is an element $x$ such that the rank of $x$ in $A$ is $\lfloor \phi n \rfloor$]%
  {phiquantile}{$\phi-quantile$}{$\phi-quantile$ of $A$}
  
  \newacronym[%
  category=notation-category,%
  description= an interval-based approach to memory management for concurrent data structures]%
  {IBR}{IBR}{Interval-Based-Reclamation}
  
    \newacronym[%
  category=notation-category,%
  description=]%
  {way merge}{K-way merge}{Merge algorithm consisting of merging K sorted arrays to produce a single sorted array with the same elements}
% --------------------------------

% Commands below will control the behavior/appearance of the list of abbreviations and acronyms

% Uncomment this command to have _all_ abbreviations and acronyms defined
% in this file appear in the final list - rather than just the ones you
% use in the thesis
\keepUnusedAbbreviations
