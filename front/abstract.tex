% This file contains the abstract part of your thesis - in English and
% in Hebrew (within \abstractEnglish and \abstractHebrew respectively).
%
% Notes:
% - This file uses the UTF-8 character set encoding for the Hebrew
%   text not to get garbled. Keep it that way.
% - Assuming your thesis is mainly in English, Graduate School 
%   regulations mandate the following lengths for the abstracts:
%
%      Language    Min. Length   Max. Length
%     ---------------------------------------
%      English       200 words     500 words
%      Hebrew        500 words   2,000 words
%
%   so that the Hebrew abstract typically has some content from
%   the English introduction and an overview of the results, not
%   present in the English; it is not just a translation.

\abstractEnglish{

Sketches are a family of streaming algorithms widely used in the world of big data to perform fast, real-time analytics. A popular sketch type is Quantiles, which estimates the data distribution of a large input stream. We present Quancurrent, a highly scalable concurrent Quantiles sketch. Quancurrent’s throughput increases linearly with the number of available threads, and with $32$ threads, it reaches an update speedup of $12$x and a query speedup of $30$x over a sequential sketch. Quancurrent allows queries to occur concurrently with updates and achieves an order of magnitude better query freshness than existing scalable solutions.

% So this should contain a few more paragraphs... we'll fill them using some placeholder text (in Latin):

% \lipsum[10-12]

} % end of English abstract


\abstractHebrew{

% Note that certain commands don't work that well in Hebrew "mode".
% If this happens to you, try wrapping the command within a
% \textenglish{ } - that may (or may not) help.

% כאן יבוא תקציר מורחב בעברית (כאשר שפת החיבור העיקרית היא אנגלית). היקף התקציר יהיה \textenglish{1000-2000} מילים. התקציר יהווה שלמות בפני עצמו ויהיה מובן לקורא בעל ידיעות כלליות בנושא.

% בית הספר ללימודי מוסמכים מנחה מספר הנחיות לגבי התקציר בעברית:
% \begin{itemize}
% \item על התקציר להיכתב במשפטים מקושרים שלמים.
% \item בדרך-כלל אין לציין בתקציר מקורות ספרותיים וציטוטים.
% \item אין להתייחס למספר של פרק, סעיף, נוסחה, ציור או טבלה שבגוף החיבור, ואין להשתמש בקיצורים, סמלים ומונחים לא מקובלים, אלא אם יש בתקציר די מקום לזיהויים.
% \end{itemize}

% לעתים יש בכל-זאת יש צורך לכלול פקודה הכוללת קישור פנימי או חיצוני בתוך התקציר העברי; במצבים כאלו כדאי דרך-כלל לעטוף את הפקודה היוצרת את הקישור בתוך פקודת \textenglish{\texttt{\textbackslash{}textenglish\{\}}} כדי למנוע כל מיני פורענויות בלתי-רצויות, כגון כישלון בהידור קובץ ה-\textenglish{PDF} או שימוש בגופן העברי באופן אשר עלול שלא להנעים לעין. לדוגמה: נניח שיש לנו צורך לצטט מקור ביבליוגרפי. אם נעשה זאת סתם-כך: \textenglish{\texttt{\textbackslash{}cite\{Hoeffding\}}}, נקבל: \cite{Hoeffding}; אם נעטוף את פקודת הציטוט, כך: \textenglish{\texttt{\textbackslash{}textenglish\{\textbackslash{}cite\{Hoeffding\}\}}}, נקבל \textenglish{\cite{Hoeffding}} (כפי שהציטוטים נראים גם בטקסט באנגלית).

% \subsection*{\texthebrew{תת-חלק בתקציר המורחב}}

% תוכן מקוצר לגבי נושא מסוים. התייחסות ל\emph{מושג} מסוים שהחיבור בוחן. וכולי וכולי.

אנאליזה בזמן אמת של זרם נתונים גדול הינה משימה הכרחית בעולם האנאליטיקות של ביג דאטה (נתוני עתק). מערכות מודרניות מטפלות כל הזמן בכמויות עצומות של מידע המגיע בקצב גבוה תוך כדי שהן מספקות ניתוחי זמן אמת עם זמני עיכוב מינימאליים. שרתי אינטרנט מתמודדים עם פטות של בתים ועליהם להיות תמיד זמינים, אינטראקטיביים ומהירים. 

בעת ניתוח מידע בקנה מידע מאסיבי, חישוב מדויק, אפילו של שאילתות בסיסיות מאוד, עשוי לדרוש משאבי מחשוב עצומים (זיכרון וזמן חישוב). זה מוביל לפגיעה ביכולתה של מערכת להתמודד עם עומסים הולכים וגדלים. דוגמאות לשאילתות שניתן לשאול על המידע שנאסף כוללות שאילתת ספירת ערכים ייחודים, חישוב שברונים ואחוזונים, זיהוי תדר תכיפות ועוד. השיטות המדויקות לחישוב שאילתות אלה לא מצליחות להתאים את עצמן לכמויות המידע הממשיכות כל הזמן לגדול. כלים המסייעים לזהות מגמות או דפוסים מרכזיים של המידע מתוך נפח גדול של נתונים גולמיים הם בעלי ערך רב. 

סקיצות הן מבני נתונים אשר מכילות מידע מקורב על זרם נתונים גדול. הן שייכות למשפחה של אלגוריתמי סטרימינג (אלגוריתמים הפועלים על זרם מידע גדול שתמיד נכנס) הנמצאים בשימוש נרחב בעולם הביג דאטה לביצוע ניתוח נתונים בקצב ובנפח גבוה. אלגוריתמים אלה מעבדים מערך נתונים מסיבי במעבר יחיד, הם מחשבים סיכומים קטנים מאוד (הנקראים סקיצות) של המידע ומאפשרים לבצע ניתוח מהיר בזמן אמת.

סוג סקיצה שימושי בעולם הביג דאטה לביצוע ניתוח מהיר בזמן היא סקיצת שברונים. שברון של זרם נתונים היא נקודת חתך מתחתיה נמצאים כל הערכים שקטנים מהערך בנקודה זו. 
לדוגמא, השברון ה-0.5 הוא החציון.
הבנת אופן התפלגות המידע היא משימה בסיסית בניהול וניתוח נתונים, המשמשת לניטור מערכות ואפליקציות. 
סקיצת שברונים מעריכה את התפלגות הנתונים של זרם גדול כך ששאילתה לחישוב שברון מחזירה אומדן של הערך המדויק. כמו סקיצות אחרות, גם סקיצת שברונים היא בגודל תת-ליניארי והאומדנים שלה כנראה נכונים בקירוב. היא מספקת קירוב בתוך גבולות שגיאה מוכחים מתמטית עם הסתברות כשל מוגבלת. היא מאפשרת להפיק תוצאות עבור שאילתות (כגון, מהו האחוזון ה- 95) מהר יותר בכמה סדרי גודל. 

בעבודה זו אנו מציגים סקיצת שברונים מקבילית וסקילבילית המתאימה למערכות המבצעות אנאליטיקות בזמן אמת תוך שמירה על גבולות שגיאה קטנים ותוצאות שאילתות מעודכנות. 

הסקיצה בעלת תפוקה הגדלה ליניארית ככל שמספר התהליכים גדל. מתקבלת האצה של פי 12 עבור פעולות עדכון (הוספת ערך לסקיצה) והאצה של פי 30 עבור ביצוע שאילתות, עם 32 תהליכים, ביחס למקרה הסדרתי. אנו מביאים ניסויים המראים כי הסקיצה מהירה יותר מפתרונות אחרים המממשים סקיצה מקבילית. כמו- כן הסקיצה בעלת תוצאות עדכניות יותר בסדר גודל מאשר הסקיצה המקבילית העדכנית ביותר שידועה לנו נכון לזמן כתיבת עבודת מחקר זו. 

הארכיטקטורה של הסקיצה המקבילית מבוססת על חציצה מקומית של ערכים, אשר מפועפעים לאחר מכן באצווה לסקיצה משותפת. הסקיצה מביאה לביטול צוואר הבקבוק שנובע בעיקר מהצורך במיון חוצצים גדולים. המיון בסקיצה המקבילית מתבצע בשלוש רמות והסקיצה המשותפת עצמה מאורגנת במספר רמות אשר עשויות להיות מופצות וממוינות במקביל. שאילתות מטופלות כל הזמן במקביל לפעולות עדכון. כדי לאפשר סקילביליות של שאילתות, הסקיצה משרתת אותן מתמונת מצב של הסקיצה המשותפת השמורה במטמון מקומי תוך שמירה על רעננות התוצאות. 

בעבודה זו אני מגדירים רשמית את מודל המערכת ומביאים הוכחה פורמאלית לנכונות הסקיצה המקבילית. 
} % end of Hebrew abstract
